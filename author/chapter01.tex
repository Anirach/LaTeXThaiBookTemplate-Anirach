\chapter{ชื่อของบท....}

\section{หัวข้อแรก..........}
เนื้อหาแรก....... \Gls{DevOps}

\begin{tcolorbox}[colframe=gray!60, colback=gray!6, title=แนวคิดสำคัญ:]
% This part is outside listing so that it won't have line numbers
\begin{itemize}
\item{ {\bf การกำหนดตัวแปร:} การกำหนดค่าให้กับตัวแปร.}
\item{ {\bf การตั้งชื่อตัวแปร:} การกำหนดชือที่เหมาะสมสำหรับตัวแปรตามรูปแบบและแนวทางที่กำหนด.}
\end{itemize}
\end{tcolorbox}


\subsection{หัวข้อย่อยที่ 1 .....ของหัวข้อแรก}
เนื้อหาย่อยของหัวข้อแรก  JavaScript \index{JavaScript}  is nested \index{JavaScript!nested}... .\textcite{flanagan2016javascript}

\subsection{หัวข้อย่อยที่ 2 .....ของหัวข้อแรก}
เนื้อหาย่อยของหัวข้อแรก


\section{หัวข้อที่ 2 ..........}
เนื้อหาแรก.......

\subsection{หัวข้อย่อยที่ 1 .....ของหัวข้อที่ 2}
เนื้อหาย่อยของหัวข้อแรก

\subsection{หัวข้อย่อยที่ 2 .....ของหัวข้อที่ 2}
เนื้อหาย่อยของหัวข้อแรก

\begin{table}[H]
\centering
\caption{ตารางแสดงข้อมูลประชากร}
\label{tab:population}
\begin{tabular}{lrr}
\toprule
\textbf{ประเทศ} & \textbf{ประชากร (ล้านคน)} & \textbf{พื้นที่ (ตร.กม.)} \\
\midrule
ประเทศไทย    & 69   & 513,120 \\
ญี่ปุ่น       & 126  & 377,975 \\
จีน           & 1,400 & 9,596,961 \\
อินเดีย       & 1,366 & 3,287,263 \\
สหรัฐอเมริกา  & 331  & 9,525,067 \\
\bottomrule
\end{tabular}
\end{table}

\section{หัวข้อที่ 3 ..........}
เนื้อหาแรก.......

\subsection{หัวข้อย่อยที่ 1 .....ของหัวข้อที่ 3}
เนื้อหาย่อยของหัวข้อแรก

\subsection{หัวข้อย่อยที่ 2 .....ของหัวข้อที่ 3}
เนื้อหาย่อยของหัวข้อแรก